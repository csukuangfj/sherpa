\subsection{提取特征}

\begin{frame}[t,fragile]
\frametitle{提取特征 --- 分帧 (1/2)}
\begin{center}
\includegraphics[width=0.45\textwidth]{pic/feature-window-size.jpg}
\end{center}
\begin{itemize}
\item 分帧
  \begin{itemize}
    \item 一般一帧长 25 ms (也叫 \texttt{frame length}), 即 400 个采样点
    \item 帧偏移 10 ms (也叫 \texttt{frame shift}), 即 160 个采样点
  \end{itemize}
\end{itemize}
\visible<2->{
  \textbf{\textcolor{red}{问题}}: 1 秒的数据, 可以分成多少帧? (采样率为 16000 Hz)
}

\visible<3->{
  \footnotesize{
    \textbf{\textcolor{red}{提示}}: Conv1d 中, Output shape 与 kernel size 及 stride 的关系 (不考虑 padding)
  }
}

\visible<4->{
\begin{equation}
\frac{16000 - 400}{160} + 1 = 98.5, \textrm{向0取整,得} 98
\end{equation}
}

\visible<5->{
  一般的,可以近似认为, 1秒钟 100 帧, 10秒钟 1000 帧
}
\end{frame}

\begin{frame}[t,fragile]
\frametitle{提取特征 --- 分帧 (2/2)}
\begin{itemize}
    \item 实战
\end{itemize}
\begin{lstlisting}[basicstyle={\tiny},language={python}]
import numpy as np
np.random.seed(202250927)

sample_rate = 16000
num_seconds = 1
samples = np.random.uniform(low=`\texttt{-1}`, high=1, size=(sample_rate * num_seconds,))

frame_length_ms = 25
frame_shift_ms = 10

frame_length = int(frame_length_ms * sample_rate / 1000)
frame_shift = int(frame_shift_ms * sample_rate / 1000)

frame_samples_0 = samples[:frame_length]
frame_samples_1 = samples[1*frame_shift:(1*frame_shift + frame_length)]
frame_samples_2 = samples[2*frame_shift:(2*frame_shift + frame_length)]
\end{lstlisting}
\visible<2->{
  \textbf{\textcolor{red}{问题}}: 第 $n$ 帧的起始时间是多少?
}

\visible<3->{
  $n * \mathrm{frame\_shift}$
}

\visible<4->{
  \begin{itemize}
    \item 对于一些模型, 利用这个信息,可以给识别结果加\textbf{\textcolor{red}{时间戳}}
  \end{itemize}
}

\end{frame}

\begin{frame}[t,fragile]
\frametitle{提取特征 --- 去掉直流偏移}
\begin{itemize}
    \item Remove DC offset
\end{itemize}
\begin{lstlisting}[language={python}]
def remove_dc_offset(samples):
  mean = np.mean(samples)
  return samples - mean
\end{lstlisting}

\visible<2->{
\begin{itemize}
    \item 可选的步骤
    \item 绝大多数模型都会选择做这个
\end{itemize}
}
\end{frame}

\begin{frame}[t,fragile]
\frametitle{提取特征 --- 预加重}
\begin{itemize}
    \item Preemphasis
\end{itemize}
\begin{lstlisting}[language={python}]
def preemphasize(samples, coeff=0.97):
  ans = np.empty_like(samples)

  ans[0] = samples[0] - coeff * samples[0]
  ans[1:] = samples[1:] - coeff * samples[:-1]

  return ans
\end{lstlisting}

\visible<2->{
\begin{itemize}
    \item 可选的步骤
    \item 绝大多数模型都会选择做这个
\end{itemize}
}
\end{frame}

\begin{frame}[t,fragile]
\frametitle{提取特征 --- 加窗}
\begin{itemize}
    \item 有非常多种窗函数
    \item \texttt{hann} 窗函数是 \texttt{PyTorch} 里默认的窗函数
        \begin{itemize}
          \item {\tiny \url{https://docs.pytorch.org/docs/stable/generated/torch.hann\_window.html}}
        \end{itemize}
\end{itemize}
\begin{equation}
  w_n = \frac{1}{2} \left(1 - \cos \left( \frac{2\pi n}{N-1} \right) \right)
\end{equation}
\begin{lstlisting}[language={python}]
def get_hann_window(n: int):
  # 请看 `{\tiny https://docs.pytorch.org/docs/stable/generated/torch.hann\_window.html}`
  k = np.arange(n)
  return 0.5 * (1 - np.cos(2 * np.pi * k / (n - 1)))

def apply_window(samples, window):
  return samples * window
\end{lstlisting}
\end{frame}

\begin{frame}[t,fragile]
\frametitle{提取特征 --- 计算 FFT}
\begin{lstlisting}[language={python}]
def compute_fft(samples, nfft=512):
    return np.fft.rfft(samples, nfft)
\end{lstlisting}
  \begin{itemize}
      \item 如果采样数是400点,后面补 112 个0, 做 512 点的 fft
      \item 返回一个\textbf{\textcolor{red}{复数}}向量,维度为 $\frac{512}{2}+1 = 257$
  \end{itemize}
\end{frame}

\begin{frame}[t,fragile]
\frametitle{提取特征 --- 计算功率谱}
\begin{lstlisting}[language={python}]
def compute_power_spectrum(fft_bins):
  return np.abs(fft_bins) ** 2
\end{lstlisting}
  \begin{itemize}
    \item 返回一个\textbf{\textcolor{red}{实数}}向量,维度为 $\frac{512}{2}+1 = 257$
  \end{itemize}
\end{frame}

\begin{frame}[t,fragile]
\frametitle{提取特征 --- 乘以 Mel filter bank 矩阵}
\begin{lstlisting}[language={python}]
pip install kaldi-native-fbank
\end{lstlisting}
\begin{lstlisting}[language={python}]
import kaldi_native_fbank as knf

def get_mel_filter_bank_matrix():
    mel_opts = knf.MelBanksOptions()
    mel_opts.num_bins = 23

    frame_opts = knf.FrameExtractionOptions()
    mel_bank = knf.MelBanks(opts=mel_opts, frame_opts=frame_opts)
    return mel_bank.get_matrix()

matrix = get_mel_filter_bank_matrix()
print(matrix.shape)
\end{lstlisting}
\only<2>{
  \begin{itemize}
    \item 假定特征维度为 23, \texttt{n\_fft = 512}
    \item \texttt{matrix} 的形状为 $(23, 257)$
  \end{itemize}
}
\only<3>{
  \begin{itemize}
    \item \texttt{matrix} 有何特点?稀疏性?
  \end{itemize}
}
\end{frame}

\begin{frame}[t,fragile]
\frametitle{提取特征 --- All in one}
\begin{lstlisting}[basicstyle={\tiny},language={python}]
def compute_fbank(samples):
    samples = remove_dc_offset(samples)
    samples = preemphasize(samples)

    window = get_hann_window(samples.shape[0])
    samples = samples * window

    fft_bins = compute_fft(samples)
    power_spec = compute_power_spectrum(fft_bins)

    matrix = get_mel_filter_bank_matrix()

    f = np.matmul(matrix, power_spec.reshape(-1, 1)).squeeze(1)

    f = np.where(f == 0, np.finfo(float).eps, f)

    return np.log(f)

feature_frame_0 = compute_fbank(frame_samples_0)
feature_frame_1 = compute_fbank(frame_samples_1)
print(feature_frame_0.shape)
print(feature_frame_1.shape)
\end{lstlisting}
  \begin{itemize}
      \item 每一帧的维度都是 23
      \item 1秒钟的特征组成一个二维矩阵, 形状为 $(98, 23)$, 作为神经网络的输入
  \end{itemize}
\end{frame}

\begin{frame}[t,fragile]
\frametitle{提取特征 --- 实战}
\begin{itemize}
\item Google colab notebook: {\tiny \url{https://colab.research.google.com/drive/1NxiyngCA4lDRy5BxqTVDLGDPudWKRwEs?usp=sharing}}
\item GitHub 备份: {\tiny \url{https://github.com/k2-fsa/colab/blob/master/fbank_demo.ipynb}}
\end{itemize}

\begin{table}
\centering
% https://qr.io
\begin{tabular}{cc}
\includegraphics[width=0.45\textwidth]{./pic/fbank-colab.png}&
\includegraphics[width=0.45\textwidth]{./pic/fbank-github.png}\\
\end{tabular}
\end{table}

\end{frame}

\begin{frame}[t,fragile]
\frametitle{提取特征 --- 总结}
\begin{enumerate}
    \item 分帧
    \item 去掉直流偏移
    \item 预加重
    \item 加窗
    \item 计算 FFT
    \item 计算功率谱
    \item 乘以 Mel filter bank 矩阵
    \item 取 \texttt{log}
\end{enumerate}
\end{frame}

\begin{frame}[t,fragile]
\frametitle{提取特征 --- 扩展}
\begin{itemize}
  \item  1 秒钟的音频,可以得到一个二维矩阵 \texttt{features},形状为 $(98, 23)$
  \item \textbf{\textcolor{red}{提问}}: 计算第 $i$ 帧特征时,是否需要\textbf{\textcolor{red}{全部}}的1秒数据 ?
\end{itemize}
\end{frame}

