% settings for TikZ

\usepackage[tightpage, active]{preview}
\setlength{\PreviewBorder}{0pt} % the space between the final figure and the border

\usepackage{amsmath}

\usepackage[table]{xcolor}

\usepackage{graphicx}
\usepackage{caption}
\usepackage{subcaption}
\usepackage{hyperref}

%===========================================
%   Tikz
%-------------------------------------------
\usepackage{tikz}
\usetikzlibrary{%
  arrows,%
  automata,%
  backgrounds,%
  calc,%
  chains,%
  decorations.pathmorphing, % for snaking lines
  decorations.pathreplacing,%
  fit,%
  matrix,%
  positioning,%
  scopes,%
  shapes.geometric,%
  shapes,%
  shapes.symbols,%
  spy,%
  trees%
}

% http://mirrors.sjtug.sjtu.edu.cn/ctan/graphics/pgf/contrib/tikz-dimline/tikz-dimline-doc.pdf
\usepackage{tikz-dimline}

\usepackage{pgfplots}
%\pgfplotsset{compat=1.13}

%===========================================
%   preview environment
%   this should be added after all packages
%-------------------------------------------
\PreviewEnvironment{forest}
\PreviewEnvironment{tikzpicture}
\PreviewEnvironment{equation}
\PreviewEnvironment{equation*}
\PreviewEnvironment{tabular}

%===========================================
%  TikZ settings 
%-------------------------------------------
\tikzstyle{my arrow}=[%
  ->,
  >=stealth',
  shorten >=1pt % if enabled, the arrow head will not touch the edges
]

\tikzstyle{my line}=[%
  % thick,
  line width=2.5pt,
]

\tikzstyle{my dotted box}=[%
  draw=black!50!white,
  line width=1pt,
  dash pattern=on 1pt off 4pt on 6pt off 4pt,
  inner sep=4mm,
  rectangle,
  rounded corners,
]

\tikzstyle{my node text}=[%
  text width=15em,
  text centered,
]

\tikzstyle{my node}=[%
  my node text,
  line width=1pt,
  anchor=center,
  rectangle, % shape of the node
  rounded corners,
  % fill=black!10,
  draw=black, % color of the edge
  very thick, % thick edges
  minimum height=3.8em,
  inner sep=0pt,  % there is no extra space around the text, as it
  % it has already specified the minimum size of the node
]

\tikzstyle{my fill gray}=[%
  fill=gray!20,
]

\tikzstyle{my fill green}=[%
  fill=green!40,
]

\tikzstyle{my fill red}=[%
  fill=red!40,
]
