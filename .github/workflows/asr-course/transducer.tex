\subsection{Transducer}

\begin{frame}[t]
\frametitle{Transducer 介绍}
\small
\begin{itemize}
\item 很有代表性的一个模型是 \textbf{英伟达} 开源的 \textbf{parakeet-tdt-0.6b-v2} 模型
  \begin{itemize}
    \item \url{https://huggingface.co/nvidia/parakeet-tdt-0.6b-v2}
  \end{itemize}
\end{itemize}

\begin{center}
\includegraphics[width=0.8\textwidth]{./pic/tdt-leaderboard.png}
\end{center}

\end{frame}


\begin{frame}[t,fragile]
\frametitle{Transducer 组成 (1/3)}
\begin{itemize}
\item 3 部分: encoder 模型, decoder 模型, joiner 模型
\end{itemize}

\begin{itemize}
\item \textbf{encoder}
\end{itemize}

\begin{lstlisting}
samples = load("cat.wav")

features = compute_features(samples)

encoder_out = encoder(features)
\end{lstlisting}

\begin{itemize}
\item \texttt{encoder\_out} 的 shape 为 \texttt{(num\_frames, encoder\_dim)}
\item 一般把这里的 \texttt{num\_frames} 叫做 \texttt{T}
\end{itemize}
\end{frame}


\begin{frame}[t,fragile]
\frametitle{Transducer 组成 (2/3)}
\begin{itemize}
\item 3 部分: encoder 模型, decoder 模型, joiner 模型
\end{itemize}

\begin{itemize}
\item \textbf{decoder}
\end{itemize}

\begin{lstlisting}
current = xx
decoder_out = decoder(current)
\end{lstlisting}

\begin{itemize}
\item \texttt{current} 为当前识别出来的 token ID
  \begin{itemize}
    \item 开始识别之前的 ID 为一个特殊的 blank ID
    \item AED 里面有 bos 和 sos
    \item transducer 里面有 blank
  \end{itemize}
\item \texttt{decoder\_out} 的 shape 为 \texttt{(1, decoder\_dim)}
\end{itemize}

\end{frame}


\begin{frame}[t,fragile]
\frametitle{Transducer 组成 (3/3)}
\begin{itemize}
\item 3 部分: encoder 模型, decoder 模型, joiner 模型
\end{itemize}

\begin{itemize}
\item \textbf{joiner}
\end{itemize}

\begin{lstlisting}
encoder_out_t = encoder_out[t:t+1]

logits = joiner(encoder_out_t, decoder_out)
\end{lstlisting}

\begin{itemize}
\item \texttt{t} 的取值从0 到 \texttt{T-1}
\item T 为 \texttt{encoder\_out.shape[0]}
\end{itemize}

\end{frame}


\begin{frame}[t,fragile]
\frametitle{Transducer 解码}
\begin{itemize}
\item 3 部分: encoder 模型, decoder 模型, joiner 模型
\end{itemize}

\begin{lstlisting}
ids = [ ]
t = 0
while t < T:
  encoder_out_t = encoder_out[t:t+1]
  logits = joiner(encoder_out_t, decoder_out)
  current = logits.argmax()
  if current == blank:
    t += 1
  else:
    ids.append(current)

    decoder_out = decoder(current)

// 查表:把  ids 变成 token
// 然后把所有 token 依次连接起来
\end{lstlisting}
\begin{itemize}
\item<2-> \textbf{\textcolor{red}{问题}} 如果一直不解码出来 blank, 该怎么办?
\end{itemize}

\end{frame}


\begin{frame}[t]
\frametitle{常用的 transducer 模型}
\begin{enumerate}
\item 很有代表性的一个模型是 \textbf{英伟达} 开源的 \textbf{parakeet-tdt-0.6b-v2} 和 \textbf{parakeet-tdt-0.6b-v3}
\item \textbf{新一代Kaldi}开源的 Zipformer transducer 模型
\end{enumerate}
\end{frame}

\begin{frame}[t]
\frametitle{transducer 解码实战}

\begin{itemize}
\item Google colab notebook: {\tiny \url{https://colab.research.google.com/drive/1l8EqNGTsdmxbG6uu1mDCBH4QpI_KKaP-?usp=sharing}}
\item GitHub 备份: {\tiny \url{https://github.com/k2-fsa/colab/blob/master/transducer_decoding_demo.ipynb}}
\end{itemize}

\begin{table}
\centering
% https://qr.io
\begin{tabular}{cc}
\includegraphics[width=0.45\textwidth]{./pic/transducer-colab.png}&
\includegraphics[width=0.45\textwidth]{./pic/transducer-github.png}\\
\end{tabular}
\end{table}

\end{frame}

