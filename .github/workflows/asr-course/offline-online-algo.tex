\subsection{离线与在线算法}

\begin{frame}[t]
\frametitle{在线算法}
\begin{center}
\includegraphics[width=1.0\textwidth]{./pic/online-algo.jpg}
\end{center}

\begin{itemize}
\item \textbf{\textcolor{red}{注意}}: \textbf{在线}算法中的在线,和\textbf{联网}没关系
\end{itemize}
\end{frame}

\begin{frame}[t]
\frametitle{离线算法}
\begin{center}
\includegraphics[width=1.0\textwidth]{./pic/offline-algo.jpg}
\end{center}

\begin{itemize}
\item \textbf{\textcolor{red}{注意}}: \textbf{离线}算法中的离线,和\textbf{联网}没关系
\end{itemize}
\end{frame}

\begin{frame}[t]
\frametitle{流式语音识别}
\begin{itemize}
\item \textbf{不}需要\textbf{所有}的数据作为输入
\item 使用流式的模型 (在线算法)
\item 边说边识别
\end{itemize}
\end{frame}

\begin{frame}[t]
\frametitle{非流式语音识别}
\begin{itemize}
\item 使用非流式的模型 (离线算法)
\item 需要等待\textbf{\textcolor{red}{所有}}的数据作为输入, 说完一句话之后,才开始识别
\end{itemize}
\end{frame}

\begin{frame}[t,fragile]
\frametitle{流式与非流式语音识别比较}
{\tiny
\begin{table}[htbp]
    \centering
    \caption{流式语音识别与非流式语音识别的对比}
    \begin{tabular}{|l|l|l|}
        \hline
        \textbf{对比维度} & \textbf{流式语音识别} & \textbf{非流式语音识别} \\ \hline
        数据处理方式 & 边采集边传输边处理,\textbf{\textcolor{red}{增量}}式处理 & 需\textbf{\textcolor{red}{等待完整}}语音数据采集后再处理 \\ \hline
        延迟特性 & \textbf{\textcolor{red}{低延迟}}(通常几百毫秒内),实时输出中间结果 & \textbf{\textcolor{red}{延迟较高}}(依赖整体语音长度),仅输出最终结果 \\ \hline
        适用场景 & 实时对话(如语音助手、会议字幕)、实时翻译 & 长语音转写(如录音文件转文字)、批量处理任务 \\ \hline
        内存占用 & 较低(仅需缓存最近帧数据) & 较高(需加载完整语音数据) \\ \hline
        典型应用 & 微信实时语音转文字、直播字幕生成 & 录音笔语音转写、长音频归档处理 \\ \hline
    \end{tabular}
    \label{tab:asr_comparison}
\end{table}
}

\visible<2->{
\begin{enumerate}
\item<2-> \textbf{\textcolor{red}{问题}} 刷榜(刷准确率排名) 一般用流式还是非流式识别?
\item<3-> \textbf{\textcolor{red}{问题}} 流式语音识别可否识别一个文件?
\item<4-> \textbf{\textcolor{red}{问题}} 非流式语音识别对一个超长文件进行识别时,如何避免 \textbf{out of memory} 错误
\item<4-> \textbf{\textcolor{red}{问题}} 非流式语音识别可否/如何和麦克风实时录音一起用?
\end{enumerate}
}
\end{frame}
