\section{关于我们}

\begin{frame}[t]
\frametitle{Kaldi之父 (1/2)}
\begin{itemize}
    \item Daniel Povey 博士毕业于英国剑桥大学
    \item 2019 年 10 月加入小米,担任集团语音首席科学家
    \item 2022 年 12 月,入选 \textbf{\textcolor{red}{IEEE Fellow}}
\end{itemize}
\begin{center}
\begin{figure}
\includegraphics[width=0.8\textwidth]{pic/dan-scholar.png}
\end{figure}
\end{center}
\end{frame}

\begin{frame}[t]
\frametitle{Kaldi之父 (2/2)}
\begin{center}
\begin{figure}
\includegraphics[width=0.8\textwidth]{pic/dan-2.jpg}
\end{figure}
\end{center}
\end{frame}

\begin{frame}[t]
\frametitle{Kaldi}
\begin{itemize}
\item \url{https://github.com/kaldi-asr/kaldi}
\end{itemize}
\only<1>{
\begin{center}
\begin{figure}
\includegraphics[width=0.8\textwidth]{pic/kaldi-asr-github.png}
\end{figure}
\end{center}
}
\only<2>{
\begin{center}
\begin{figure}
\includegraphics[width=0.8\textwidth]{pic/kaldi-asr-github-2.png}
\end{figure}
\end{center}
}
\end{frame}

\begin{frame}[t]
\frametitle{新一代Kaldi 团队}
\begin{center}
\begin{figure}
\includegraphics[width=0.8\textwidth]{pic/k2-team.jpg}
\end{figure}
\end{center}
\end{frame}

\begin{frame}[t]
\frametitle{新一代 Kaldi 社区}
\begin{itemize}
\item 我们有\alert{\textbf{非常活跃}}的\alert{\textbf{社区}}
\end{itemize}
\begin{table}
\begin{tabular}{cccc}
\includegraphics[width=0.20\textwidth]{./pic/gongzhonghao-qr-code.jpg}&
\includegraphics[width=0.20\textwidth]{./pic/group9-qr-code.jpg}&
\includegraphics[width=0.20\textwidth]{./pic/pkufool-qr-code.jpg}&
\includegraphics[width=0.20\textwidth]{./pic/qq-group-qr-code.jpg}\\
  {\tiny 新一代 Kaldi 微信公众号}& {\tiny 微信交流 \textbf{\textcolor{red}{9}} 群}&
{\tiny 团队成员微信}& {\tiny QQ 交流群(群号: 744602236)}\\
\end{tabular}
\end{table}
\begin{itemize}
\item 微信群超过200人时,需要邀请进群
\end{itemize}
\end{frame}


\begin{frame}[t]
\frametitle{新一代Kaldi 项目}
\begin{itemize}
\item 核心算法 (\textbf{\textcolor{red}{k2}})
  \begin{itemize}
    \item \url{https://github.com/k2-fsa/k2}
    \item 含义: kaldi2, 世界第二高峰
    \item 基于 C++ 和 CUDA
  \end{itemize}

\item 数据处理 (\textbf{\textcolor{red}{lhotse}})
  \begin{itemize}
    \item \url{https://github.com/lhotse-speech/lhotse}
    \item 含义: 世界第四高峰
  \end{itemize}

\item 训练 (\textbf{\textcolor{red}{icefall}})
  \begin{itemize}
    \item \url{https://github.com/k2-fsa/icefall}
    \item 基于PyTorch
  \end{itemize}

\item 部署 (\textbf{\textcolor{red}{sherpa}})
  \begin{itemize}
    \item 基于 \textbf{torchscript}: \url{https://github.com/k2-fsa/sherpa}
    \item 基于 \textbf{onnxruntime}: \url{https://github.com/k2-fsa/sherpa-onnx}
    \item 基于 \textbf{ncnn}: \url{https://github.com/k2-fsa/sherpa-ncnn}
    \item 含义: 夏尔巴人, 专业的登山向导
  \end{itemize}

\end{itemize}
\end{frame}

\begin{frame}[t]
\frametitle{新一代Kaldi 论文}
\tiny
  Zhu H, et al. ZipVoice-Dialog: Non-Autoregressive Spoken Dialogue Generation with Flow Matching. arXiv preprint arXiv:2507.09318, 2025.

  Zhu H, et al. ZipVoice: Fast and High-Quality Zero-Shot Text-to-Speech with Flow Matching. arXiv preprint arXiv:2506.13053, 2025. \textbf{\textcolor{red}{IEEE ASRU 2025}}

  Yang Y, et al. k2SSL: A faster and better framework for self-supervised speech representation learning. arXiv preprint arXiv:2411.17100, 2024. \textbf{\textcolor{red}{ICME 2025}}

  Yao Z, et al. \textbf{\large \textcolor{red}{CR-CTC}}: Consistency regularization on CTC for improved speech recognition[J]. arXiv preprint arXiv:2410.05101, 2024. {\large \textbf{\textcolor{red}{ICLR 2025}}}

  Jin Z, et al. LibriheavyMix: a 20,000-hour dataset for single-channel reverberant multi-talker speech separation, ASR and speaker diarization[J]. arXiv preprint arXiv:2409.00819, 2024. \textbf{\textcolor{red}{INTERSPEECH 2024}}

  Kang W, et al. Libriheavy: A 50,000 hours ASR corpus with punctuation casing and context[C]//ICASSP 2024-2024 IEEE International Conference on Acoustics, Speech and Signal Processing (ICASSP). IEEE, 2024: 10991-10995. \textbf{\textcolor{red}{ICASSP 2024}}

  Yang, Xiaoyu, et al. "PromptASR for contextualized ASR with controllable style." ICASSP 2024-2024 IEEE International Conference on Acoustics, Speech and Signal Processing (ICASSP). IEEE, 2024. \textbf{\textcolor{red}{ICASSP 2024}}

  Yao, Zengwei, et al. "\textbf{\large \textcolor{red}{Zipformer}}: A faster and better encoder for automatic speech recognition." arXiv preprint arXiv:2310.11230 (2023). {\large \textbf{\textcolor{red}{ICLR 2024 (Oral)}}}

  Kang, Wei, et al. "Delay-penalized transducer for low-latency streaming ASR." ICASSP 2023-2023 IEEE International Conference on Acoustics, Speech and Signal Processing (ICASSP). IEEE, 2023. \textbf{\textcolor{red}{ICASSP 2023}}

  Kang, Wei, et al. "Fast and parallel decoding for transducer." ICASSP 2023-2023 IEEE International Conference on Acoustics, Speech and Signal Processing (ICASSP). IEEE, 2023. \textbf{\textcolor{red}{ICASSP 2023}}

  Guo, Liyong, et al. "Predicting multi-codebook vector quantization indexes for knowledge distillation." ICASSP 2023-2023 IEEE International Conference on Acoustics, Speech and Signal Processing (ICASSP). IEEE, 2023. \textbf{\textcolor{red}{ICASSP 2023}}

  Yang, Yifan, et al. "Blank-regularized ctc for frame skipping in neural transducer." arXiv preprint arXiv:2305.11558 (2023). \textbf{\textcolor{red}{INTERSPEECH 2023}}

  Yao, Zengwei, et al. "Delay-penalized CTC implemented based on Finite State Transducer." arXiv preprint arXiv:2305.11539 (2023). \textbf{\textcolor{red}{INTERSPEECH 2023}}

  Kuang, Fangjun, et al. "Pruned RNN-T for fast, memory-efficient ASR training." arXiv preprint arXiv:2206.13236 (2022). \textbf{\textcolor{red}{INTERSPEECH 2022}}

\end{frame}

\begin{frame}[t]
\frametitle{新一代Kaldi Interspeech 2021 tutorial}
\begin{center}
  \includegraphics[width=0.8\textwidth]{./pic/next-gen-interspeech-2021.jpg}
\end{center}

\end{frame}

\begin{frame}[t]
\frametitle{新一代Kaldi Interspeech 2023 tutorial}
\begin{center}
  \includegraphics[width=0.8\textwidth]{./pic/next-gen-kaldi-2023.jpg}
\end{center}

\end{frame}

\begin{frame}[t]
  \frametitle{新一代Kaldi Job (1/2)}
  \begin{itemize}
      \item We are still in Hiring: Both intern and full-time
  \end{itemize}

\begin{center}
  \includegraphics[width=0.4\textwidth]{./pic/liliana.jpg}
\end{center}
\end{frame}

\begin{frame}[t]
  \frametitle{新一代Kaldi Job (2/2)}
\begin{center}
  \includegraphics[width=0.8\textwidth]{./pic/job.jpg}
\end{center}
\end{frame}

\begin{frame}[t]
\frametitle{个人介绍}
\begin{itemize}
\item 中南大学, 本科, 自动化专业
\item 德国斯图加特大学, 硕士,信息技术专业
\item 现为小米算法工程师,新一代 Kaldi 团队成员
\item \textbf{\textcolor{red}{GitHub}} 主页: \url{https://github.com/csukuangfj}
\end{itemize}
\begin{center}
\includegraphics[width=0.8\textwidth]{pic/my-github.jpg}
\end{center}
\end{frame}
