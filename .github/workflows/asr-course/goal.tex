\section{课程介绍}
\begin{frame}[t]
\frametitle{学生需求分析}
\begin{itemize}
\item<1-> 学习前端知识(本课程\textbf{没有涉及})
\item<2-> 移动端开发知识(本课程\textbf{没有涉及})
\item<3-> 嵌入式端开发知识(本课程\textbf{没有涉及})
\item<4-> 与实际应用有关,不拘泥于理论
    \begin{itemize}
        \item 使用已有的工具
        \item 使用 API,进行二次开发
    \end{itemize}
\item<5-> 学习新一代 \texttt{Kaldi} 语音识别技术
\end{itemize}
\end{frame}

\begin{frame}[t]
\frametitle{基本原则}
\begin{itemize}
\item<1-> 多一道公式,少一半听众 (尽量做到通俗易懂)
\item<2-> Talk is cheap, show me the code
\item<3-> 可以随时提出问题
    \begin{itemize}
      \item 回答问题就有\textbf{\large \textcolor{red}{奖}}
      \item 美好本 (\xout{10}\ \ 9 本)
         \item \xout{新一代 Kaldi \textbf{新款} T 恤 (4 件)}
            \begin{itemize}
                \item \xout{L 一件, XL 两件, 2 XL 一件}
            \end{itemize}
          \item \xout{每人限领一件或者一本 (2选1)}
         \item \textbf{\textcolor{red}{数量有限、领完为止}}
    \end{itemize}
\item<4-> 不需要同学们自己训模型(鼓励有机器资源的同学动手跑一遍)
\end{itemize}
\end{frame}

\begin{frame}[t]
\frametitle{教学目的}
\begin{itemize}
  \item 4 小节课的学习 (9.27, 10.13)
\item 了解语音识别的基本原理
  \begin{itemize}
    \item 语音特征的计算
    \item 语音活动检测 (Voice activity detection, VAD)
    \item 3 种端到端语音识别的解码方法
  \end{itemize}

\item 语音识别实战
  \begin{itemize}
    \item \sout{新一代 \texttt{Kaldi} 部署框架安装和使用}
    \item \sout{电脑端、 移动端、嵌入式端、网页端等本地语音识别}
    \item 生成字幕实战
  \end{itemize}
\end{itemize}
\end{frame}

\begin{frame}[t]
\frametitle{现场演示}
\begin{enumerate}
\item 中文视频和字幕
    \begin{itemize}
        \item 雷军年度演讲
    \end{itemize}

\item 英文视频和字幕
    \begin{itemize}
        \item Obama 演讲
    \end{itemize}

\item 去掉音频中的非人声
    \begin{itemize}
        \item 演示 VAD 的作用
    \end{itemize}
\end{enumerate}

\begin{itemize}
  \item \textbf{\textcolor{red}{备注}}: 使用目前开源的 state-of-the-art 模型得到的结果
  \item 本地运行, 不需要访问服务器
  \item 不需要 GPU
  \item 只需个人电脑 CPU
\end{itemize}

\end{frame}

\begin{frame}[t,fragile]
\frametitle{授课目标}
\begin{itemize}
\item<1-> 现在: \textbf{\textcolor{red}{Unknown unknowns}} (about speech recognition)
\item<2-> 学完这门课后: \textbf{\textcolor{red}{Known unknowns}} (about speech recognition)
\item<3-> 课后自行努力学习后: \textbf{\textcolor{red}{Known knowns}} (about speech recognition)
\end{itemize}

\vspace{4cm}
{\footnotesize
\visible<3->{来源: \url{https://www.danielpovey.com/files/Lecture1.pdf}}
}
\end{frame}
